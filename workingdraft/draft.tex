%  LaTeX support: latex@mdpi.com 
%  In case you need support, please attach all files that are necessary for compiling as well as the log file, and specify the details of your LaTeX setup (which operating system and LaTeX version / tools you are using).

%=================================================================
\documentclass[geosciences,article,submit,moreauthors,pdftex]{Definitions/mdpi} 

% If you would like to post an early version of this manuscript as a preprint, you may use preprint as the journal and change 'submit' to 'accept'. The document class line would be, e.g., \documentclass[preprints,article,accept,moreauthors,pdftex]{mdpi}. This is especially recommended for submission to arXiv, where line numbers should be removed before posting. For preprints.org, the editorial staff will make this change immediately prior to posting.

%---------
% article
%---------
% The default type of manuscript is "article", but can be replaced by: 
% abstract, addendum, article, benchmark, book, bookreview, briefreport, casereport, changes, comment, commentary, communication, conceptpaper, conferenceproceedings, correction, conferencereport, expressionofconcern, extendedabstract, meetingreport, creative, datadescriptor, discussion, editorial, essay, erratum, hypothesis, interestingimages, letter, meetingreport, newbookreceived, obituary, opinion, projectreport, reply, retraction, review, perspective, protocol, shortnote, supfile, technicalnote, viewpoint
% supfile = supplementary materials

%----------
% submit
%----------
% The class option "submit" will be changed to "accept" by the Editorial Office when the paper is accepted. This will only make changes to the frontpage (e.g., the logo of the journal will get visible), the headings, and the copyright information. Also, line numbering will be removed. Journal info and pagination for accepted papers will also be assigned by the Editorial Office.

%------------------
% moreauthors
%------------------
% If there is only one author the class option oneauthor should be used. Otherwise use the class option moreauthors.

%---------
% pdftex
%---------
% The option pdftex is for use with pdfLaTeX. If eps figures are used, remove the option pdftex and use LaTeX and dvi2pdf.

%=================================================================
\firstpage{1} 
\makeatletter 
\setcounter{page}{\@firstpage} 
\makeatother
\pubvolume{xx}
\issuenum{y}
\articlenumber{z}
\pubyear{2020}
\copyrightyear{2020}
\externaleditor{Academic Editor: Dr. Rafael O. Tinoco}
\history{Received: date; Accepted: date; Published: date}
%\updates{yes} % If there is an update available, un-comment this line

%% MDPI internal command: uncomment if new journal that already uses continuous page numbers 
%\continuouspages{yes}

%------------------------------------------------------------------
% The following line should be uncommented if the LaTeX file is uploaded to arXiv.org
%\pdfoutput=1

%=================================================================
% Add packages and commands here. The following packages are loaded in our class file: fontenc, calc, indentfirst, fancyhdr, graphicx, lastpage, ifthen, lineno, float, amsmath, setspace, enumitem, mathpazo, booktabs, titlesec, etoolbox, amsthm, hyphenat, natbib, hyperref, footmisc, geometry, caption, url, mdframed, tabto, soul, multirow, microtype, tikz

\newcommand\Rey{\mathrm{Re}}
\usepackage{siunitx}

%=================================================================
%% Please use the following mathematics environments: Theorem, Lemma, Corollary, Proposition, Characterization, Property, Problem, Example, ExamplesandDefinitions, Hypothesis, Remark, Definition, Notation, Assumption
%% For proofs, please use the proof environment (the amsthm package is loaded by the MDPI class).

%=================================================================
% Full title of the paper (Capitalized)
\Title{Effective collector efficiency of sparse and dense arrays of emergent cylindrical collectors in laminar-turbulent transitional flows}

% Author Orchid ID: enter ID or remove command
\newcommand{\orcidauthorA}{0000-0002-7970-841X} % Add \orcidA{} behind the author's name
%\newcommand{\orcidauthorB}{0000-0000-000-000X} % Add \orcidB{} behind the author's name

% Authors, for the paper (add full first names)
\Author{Jordan Wingenroth $^{1}$*\orcidA{}, Candace Yee $^{2}$, Justin Nghiem$^{1,3}\dagger$ and Laurel Larsen $^{1,2}$}

% Authors, for metadata in PDF
\AuthorNames{Jordan Wingenroth, Candace Yee, Justin Nghiem and Laurel Larsen}

% Affiliations / Addresses (Add [1] after \address if there is only one affiliation.)
\address{%
$^{1}$ \quad Department of Geography, University of California, Berkeley, CA

$^{2}$ \quad Department of Civil and Environmental Engineering, University of California, Berkeley, CA

$^{3}$ \quad Department of Statistics, University of California, Berkeley, CA}

% Contact information of the corresponding author
\corres{Correspondence: j.wingenroth@berkeley.edu}

% Current address and/or shared authorship
\firstnote{Current address: California Institute of Technology, Pasadena, CA} 
%\secondnote{These authors contributed equally to this work.}
% The commands \thirdnote{} till \eighthnote{} are available for further notes

%\simplesumm{} % Simple summary

%\conference{} % An extended version of a conference paper

%TODO
% Abstract (Do not insert blank lines, i.e. \\) 
\abstract{TO DO}

% Keywords
\keyword{sediment transport; collector efficiency; transitional turbulence}

\begin{document}

\section{Introduction}

\cite{Fauria_2015}

\subsection{Environmental Significance}

\subsection{Theoretical Background}

\section{Materials and Methods}

\subsection{Experimental Methods}

\subsection{Sediment Transport and Particle Capture Model}

\section{Results}

\subsection{Effects of Collector Density and Reynolds Number}

Effective collector efficiency was found to decrease with increasing collector Reynolds number and collector density in nearly all cases \ref{figXXX}. One exception occurred at the lowest collector density ($\phi = 0.226\%$), where the intermediate $\Rey_c$ treatment ($\Rey_c = 133$) had lower \%ECE than both the lower and higher $\Rey_c$ treatments. When compared to the other collector density treatments at this $\Rey_c = 133$ level, this treatment's \%ECE was slightly lower than the intermediate collector-density treatment, and was greater than the highest collector-density treatment. Repeating the suspended sediment experiment for this combination of parameter values yielded a \%ECE value very close to the estimate from the original experiment.

Monte Carlo analysis determined that there is a significant negative relationship between \%ECE and collector density for the low ($\Rey_c = 66$) and high ($\Rey_c = 200$) collector Reynolds number treatments, but the trend was insignificant for the intermediate ($\Rey_c = 133$) treatment \ref{figYYYa}. This approach also found a significant negative relationship between \%ECE and $\Rey_c$ for the low ($\phi = 0.226\%$) and high ($\phi = 1.17\%$) collector density treatments, but an insignificant trend for the intermediate ($\phi = 0.650\%$) collector density treatment \ref{figYYYb}. Leaving out the insignificant relationships, the two relationships between \%ECE and collector density were considerably more significant than those between \%ECE and Reynolds number.

\subsection{Effect of Turbulence}

\begin{table}[H]
\caption{Turbulence and flow characteristics for all collector solid volume fraction ($\phi$) and $\Rey_c$ treatments, including the control (zero-collector) treatments for reference}
\centering
%% \tablesize{} %% You can specify the fontsize here, e.g., \tablesize{\footnotesize}. If commented out \small will be used.
\begin{tabular}{>{\bfseries}r>{\bfseries}rrrr}
\toprule
\textbf{$\phi$}&\textbf{$\Rey_c$}&\textbf{Turbulent Kinetic Energy (\SI{}{\metre^2/\second^2})}&\textbf{Bed Shear Stress (\SI{}{\pascal})}&\textbf{Longitudinal Velocity (\SI{}{\centi\metre/\second})}\\
\midrule
Control &   67  & \num{2.48d-6}&\num{-5.52d-4}&   2.03\\
        &   134 & \num{2.98d-6}&\num{ 3.17d-3}&   4.10\\
        &   200 & \num{3.37d-6}&\num{ 5.13d-3}&   6.24\\
\midrule
0.226\% &   67  & \num{1.23d-6}&\num{-2.95d-5}&   1.87\\
        &   134 & \num{5.26d-6}&\num{-1.39d-4}&   3.68\\
        &   200 & \num{1.22d-5}&\num{ 1.87d-2}&   5.65\\
\midrule
0.650\% &   67  & \num{9.13d-6}&\num{ 9.62d-4}&   1.39\\
        &   134 & \num{3.59d-5}&\num{ 1.20d-3}&   3.01\\
        &   200 & \num{5.96d-5}&\num{ 6.52d-4}&   4.70\\
\midrule
1.17\%  &   67  & \num{1.11d-5}&\num{ 4.47d-4}&   1.76\\
        &   134 & \num{2.98d-5}&\num{ 1.36d-3}&   4.22\\
        &   200 & \num{5.44d-5}&\num{ 2.44d-3}&   5.41\\
\bottomrule
\end{tabular}
\end{table}


\subsection{Effect of Biofilm}

\subsection{Comparison to Previous Models of Capture Efficiency}

\section{Discussion}

\subsection{Inferred Mechanisms of Effect for Collector Density and Reynolds Number}

\subsection{Relative Importance of Biofilm}

\subsection{Designing Wetlands for Maximum Sedimentation}

\vspace{6pt} 

%%%%%%%%%%%%%%%%%%%%%%%%%%%%%%%%%%%%%%%%%%
%% optional
%\supplementary{The following are available online at \linksupplementary{s1}, Figure S1: title, Table S1: title, Video S1: title.}

% Only for the journal Methods and Protocols:
% If you wish to submit a video article, please do so with any other supplementary material.
% \supplementary{The following are available at \linksupplementary{s1}, Figure S1: title, Table S1: title, Video S1: title. A supporting video article is available at doi: link.}

\authorcontributions{Conceptualization, L.L., J.N., and J.W.; methodology, L.L. and J.W.; software, L.L., J.N., J.W., and C.Y.; validation, J.W.; formal analysis, J.W. and C.Y.; investigation, J.N., J.W, and C.Y.; resources, L.L.; data curation, J.W.; writing--original draft preparation, J.N., J.W., and C.Y.; writing--review and editing, L.L., J.N., J.W., and C.Y.; visualization, J.W.; supervision, L.L., J.W., and C.Y.; project administration, L.L. and J.W.; funding acquisition, L.L.}

\funding{This research was funded by NSF award \#1455362. The APC was funded by %TODO
\textcolor{red}{WHAT GRANT WILL THE APC BE FUNDED BY}.}

%TODO
\acknowledgments{Colin Keating, Yayla Sezinger, Elle Chen, Nicole Ulakovic, Katrina Ginsberg, Danielle Satin, Sam Stein, and Sheila Trampush.}

\conflictsofinterest{The authors declare no conflict of interest.} 

%% optional
% \abbreviations{The following abbreviations are used in this manuscript:\\

% \noindent 
% \begin{tabular}{@{}ll}
% MDPI & Multidisciplinary Digital Publishing Institute\\
% DOAJ & Directory of open access journals\\
% TLA & Three letter acronym\\
% LD & linear dichroism
% \end{tabular}}

%%%%%%%%%%%%%%%%%%%%%%%%%%%%%%%%%%%%%%%%%%
%% optional
% \appendixtitles{no} %Leave argument "no" if all appendix headings stay EMPTY (then no dot is printed after "Appendix A"). If the appendix sections contain a heading then change the argument to "yes".
% \appendix
% \section{}
% \unskip
% \subsection{}
% The appendix is an optional section that can contain details and data supplemental to the main text. For example, explanations of experimental details that would disrupt the flow of the main text, but nonetheless remain crucial to understanding and reproducing the research shown; figures of replicates for experiments of which representative data is shown in the main text can be added here if brief, or as Supplementary data. Mathematical proofs of results not central to the paper can be added as an appendix.

% \section{}
% All appendix sections must be cited in the main text. In the appendixes, Figures, Tables, etc. should be labeled starting with `A', e.g., Figure A1, Figure A2, etc. 

%%%%%%%%%%%%%%%%%%%%%%%%%%%%%%%%%%%%%%%%%%
%TODO
% Citations and References in Supplementary files are permitted provided that they also appear in the reference list here. 

%=====================================
% The following MDPI journals use author-date citation: Arts, Econometrics, Economies, Genealogy, Humanities, IJFS, JRFM, Laws, Religions, Risks, Social Sciences. For those journals, please follow the formatting guidelines on http://www.mdpi.com/authors/references
% To cite two works by the same author: \citeauthor{ref-journal-1a} (\citeyear{ref-journal-1a}, \citeyear{ref-journal-1b}). This produces: Whittaker (1967, 1975)
% To cite two works by the same author with specific pages: \citeauthor{ref-journal-3a} (\citeyear{ref-journal-3a}, p. 328; \citeyear{ref-journal-3b}, p.475). This produces: Wong (1999, p. 328; 2000, p. 475)

% =====================================
% References, variant B: external bibliography
% =====================================

\reftitle{References}
\externalbibliography{yes}
\bibliography{refs}

%%%%%%%%%%%%%%%%%%%%%%%%%%%%%%%%%%%%%%%%%%
%% optional
%\sampleavailability{Samples of the compounds ...... are available from the authors.}

%% for journal Sci
%\reviewreports{\\
%Reviewer 1 comments and authors’ response\\
%Reviewer 2 comments and authors’ response\\
%Reviewer 3 comments and authors’ response
%}

%%%%%%%%%%%%%%%%%%%%%%%%%%%%%%%%%%%%%%%%%%
\end{document}

