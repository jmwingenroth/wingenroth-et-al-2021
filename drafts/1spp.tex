\documentclass{article}

\usepackage[utf8]{inputenc}

\usepackage{amsmath}
\newcommand\Rey{\mathrm{Re}}

\usepackage{amsfonts}
\usepackage{amssymb}
\usepackage{outlines}
\usepackage{xcolor}
\usepackage{geometry}
\usepackage{siunitx}

\usepackage{marginnote}
\renewcommand*{\marginfont}{\color{red}\sffamily}

\usepackage{lineno}
\renewcommand\linenumberfont{\normalfont\small\sffamily}
%\linenumbers

\usepackage{graphicx}
\graphicspath{{../pics/}}

\usepackage{natbib}

\begin{document}

\section{Introduction}

\begin{itemize}
    \item Riverine and estuarine sediment budgets play an important role in shaping aquatic and riparian geomorphology and plant communities, particularly in low-gradient coastal landforms (e.g., river deltas).
    \item In turn, plants can affect patterns of erosion and deposition via several mechanisms, one of the less well-understood being direct particle capture by submerged stems, leaves, and other surfaces. 
    \item Many previous studies have used laboratory flume experiments to examine the effects of different flow and vegetation conditions on the rate and efficiency of sediment removal from the water column.
    \item Researchers have shown that vegetation stem dimensions, spatial density, and Reynolds number are important predictors of sediment transport and deposition behavior, with analytic expressions derived from fluid dynamics principles informing the case of creeping flow, but no consensus on the nature of these relationships in transitionally turbulent flows. 
    \item Turbulence affects particle capture rates and efficiency, and is also theorized and empirically confirmed to affect other elements of sediment transport behavior, with one particularly important example being its effect on settling velocity.
    \item \cite{Palmer_2004} introduced a power law expression for estimating capture efficiency of cylindrical collectors in transitional turbulence, which takes the form 
        \begin{equation}
            \eta=C{\Rey_c}^{a}R^{b}\,.
            \label{eq:powerlaw}
        \end{equation}
    \item \cite{Fauria_2015} used the same equation form but estimated coefficients using an array of many collectors as opposed to a single one, thereby arriving at a negative relationship between $\eta^\prime$ and $\Rey$ in contrast to the positive relationship between $eta$ and $\Rey$ found by \cite{Palmer_2004}; this is a major motivator for our study. 
    \item Other motivators include expanding the parameter space to lower collector density, incorporating direct measurement of settling rates to refine capture estimates, and testing replicability and robustness of previous results across different laboratory environments.
    \item \textcolor{violet}{Do you think we should add a paragraph formally stating "we hypothesize...", or are the prior two paragraphs on motivation sufficient?}
\end{itemize}

\section{Methods}

\subsection{Suspended Concentration Model}

\subsection{Experimental Methods}

\subsubsection{Materials}

\subsubsection{Time-Decay Run Protocols}

\subsubsection{Turbulence Estimate Protocols}

\subsubsection{Flume Volume Estimate Protocol}

\subsubsection{Biofouled Runs}

\begin{itemize}
    \item Flume volume estimate
    \item Biofilm runs
\end{itemize}

\subsection{Statistical Analyses}

\subsubsection{Settling and Capture}

\subsubsection{Turbulence and Bed Shear Stress}

\section{Results}

\subsection{Settling and Collection Rates}

\subsection{Turbulence}

\section{Discussion}

\begin{itemize}
    \item \textcolor{violet}{It's important to bear in mind that, in agreement with the power law regression ($\eta^\prime = C\Rey^{-1.14}R^{0.65}$) presented by \cite{Fauria_2015}, capture efficiency ($\eta$) has an approximately inverse relationship with Reynolds number, which is in turn directly proportional to velocity ($u$). Looking at the decay model, $\eta \propto k_c/u$, which means that if $k_c$ were constant across treatments, we'd still expect the relationship $\eta \propto Re^{-1}$. Collector density shares the same proportionality ($\eta \propto k_c/I_c$). Based on our error propogation, the uncertainty in our measurements far outweighs any pattern among our $k_c$ estimates. Most of the magnitude of the observed negative trend in effective collector efficiency, over both $\Rey$ and $I_c$, is an artifact of the algebra involved in its calculation rather than actual measured differences between our treatments. [Note: I think it's important for everyone to be aware of this...I'm surprised it wasn't discussed in \cite{Fauria_2015}. It seems like such a simple but important fact]}
\end{itemize}

\bibliographystyle{apalike}
\bibliography{refs}

\end{document}
