% This file should be compiled as such: PdfLatex -> Bibtex -> PdfLatex (x2)

\documentclass{scrreprt}

\KOMAoptions{numbers=noendperiod}

\usepackage{chngcntr}
\counterwithout{figure}{chapter}
\counterwithout{equation}{chapter}

\usepackage[utf8]{inputenc}

\usepackage{amsmath}
\newcommand\Rey{\mathrm{Re}}

\usepackage{amsfonts}
\usepackage{amssymb}
\usepackage{outlines}
\usepackage{xcolor}
\usepackage{geometry}
\usepackage{siunitx}

\usepackage{marginnote}
\renewcommand*{\marginfont}{\color{red}\sffamily}

\usepackage{lineno}
\renewcommand\linenumberfont{\normalfont\small\sffamily}
\linenumbers

\usepackage{graphicx}
\graphicspath{{../pics/}}

\usepackage{natbib}

\author{Jordan Wingenroth, Candace Yee, Justin Nghiem, and Laurel Larsen}

\title{Effective collector efficiency of sparse and dense arrays of emergent cylindrical collectors in laminar-turbulent transitional flows}

\begin{document}

\maketitle

%\chapter{Introduction}
%
%TODO

\setcounter{chapter}{1}

\chapter{Methods}

\section{Theoretical Background}

In this study, we focus on direct interception of suspended particles by emergent vegetation (i.e., collectors). Direct interception is the adhesion of particles traveling along streamlines to the surfaces of collectors (e.g., plants). This is just one of several mechanisms that can remove particles from suspension. Others include gravitational settling, inertial impaction, and diffusional deposition. Gravitational settling is the process of particles denser than water being drawn towards and coming to rest on the channel bottom. Inertial impaction and diffusional deposition are other means by which suspended particles can adhere to collectors. Inertial impaction occurs when particles possess momentum relative to streamlines, and diffusional deposition arises due to some combination of turbulence and Brownian motion.

Capture efficiency ($\eta=\frac{b}{d_c}$) is defined as the ratio between the upstream width ($b$) of the streamlines that encounter a collector and the collector's own diameter ($d_c$) (Figure \ref{fig:capeff}). Effective capture efficiency ($\eta^\prime=p_r\eta$) takes into account the fact that not all particles coming into contact with a collector will adhere to it, thus including probability of retention ($p_r$) as a multiplicand in its calculation. This parameter, as opposed to theoretical capture efficiency $\eta$, corresponds to the empirical estimates made by our study and presumably any others that involve $p_r$ as an implicit factor.

\begin{figure}[htbp]
\includegraphics[width=10cm]{collectorefficiency.png}
\centering
\caption{A diagram illustrating capture efficiency for a cylindrical collector. $b$ is the horizontal width of upstream flow and $d_c$ is collector diameter. Adapted from \cite{Palmer_2004}.}
\label{fig:capeff}
\end{figure}

Particle capture in transitionally turbulent flows ($1<\Rey<1000$), which occur commonly in the environments and on the length scales of aquatic macrovegetation, does not follow the analytical expressions involving Reynolds number, collector diameter, and particle diameter ($d_p$), derived for creeping flow. Hence, empirical estimates must be used. \cite{Palmer_2004} introduced a power law expression for estimating capture efficiency of cylindrical collectors, which takes the form 
\begin{equation}
    \eta=C{\Rey_c}^{a}R^{b}\,,
    \label{eq:powerlaw}
\end{equation}
where collector Reynolds number $\Rey_c=\frac{ud_c}{\nu}$, $u$ is flow velocity, $\nu$ is kinematic velocity, and $R=\frac{d_c}{d_p}$ is the ratio between collector diameter and particle diameter ($d_p$). $C$, $a$, and $b$ are empirically determined regression coefficients. It is worth noting that \cite{Palmer_2004} used $\eta$ in their notation for this model as opposed to $\eta^\prime$ because their experiment involved an isolated single cylinder, coated with grease for which retention was assumed to be complete ($p_r = 1$).

Turbulence is also theorized and empirically confirmed to affect the settling velocity of heavy, fine particles \citep{Nielsen_1993, Jacobs_2016, Wang_2018}. Increasing turbulence is one mechanism by which collectors could affect settling rates.

\section{Suspended Particle Concentration Model}

We used a model adapted from \cite{Fauria_2015} to estimate collection efficiency based on exponential decay of suspended sediment concentration. The key equation in the exponential decay model is
\begin{equation}
    \frac{d\bar{\phi}}{dt} = -[\frac{Cv_s}{h}(1-E_r) + \eta^{\prime}ud_cI_c]\bar{\phi}(t) = -k\bar{\phi}(t)\,.
    \label{eq:model}    
\end{equation}
The left and right summands in brackets represent the decay rate of depth-averaged suspended sediment concentration ($\bar{\phi} = \frac{1}{h} \int_0^h\phi(z)dz$) due to settling ($k_s$) and collection ($k_c$) respectively, where $z$ represents position along the vertical axis. $k = k_s + k_c$ is the total decay rate of suspended sediment.  $C$ is a constant based on the shape of the vertical sediment profile, which is assumed to follow the Rouse equation; $v_s$ is the settling velocity; $h$ is water depth; and $E_r$ is the rate of entrainment (i.e., uptake of particle from the channel bed). Please refer to \cite{Fauria_2015} for a detailed derivation of the formula for $k_s$ from established fluid dynamics principles. As for $k_c$, it is derived from flow velocity ($u$), previously described properties of collectors ($\eta^\prime$ and $d_c$), and collector density ($I_c = \frac{N_ch}{V}$), where $N_c$ is number of collectors and $V$ is the fluid volume (i.e., channel bed area times water depth). The volumetric rate at which water passes through a lateral cross-section with area equal to the frontal area of one collector is calculated as $ud_ch$. The number of particles passing through this area is found by multiplying by depth-averaged concentration ($N_p = ud_ch\bar{\phi}$). The rate at which particles are removed from suspension by this single collector is $\frac{dN_p}{dt}=\eta^{\prime}ud_ch\bar{\phi}$. By scaling up to the suite of collectors, modeled identically to one another, the change in concentration due to collection can be derived:
\begin{equation}
\frac{d\bar{\phi}}{dt} = \frac{N_c}{V}\frac{dN_p}{dt} = \frac{N_c}{V}\eta^{\prime}ud_ch\bar{\phi} = \eta^{\prime}ud_cI_c\bar{\phi}\,.
\label{eq:collection}
\end{equation}

Because we assume the vertical profile of sediment concentration maintains a constant shape through time, governed by the Rouse equation, concentration at any height ($\phi_a$) can be assumed to be proportional to average concentration ($\phi_a=\frac{\bar{\phi}}{C}$). The solution of the above differential equation ($\bar{\phi}(t) = \bar{\phi}(0)e^{-kt}$) can thus be freely interchanged with an equation of identical form for concentration at a specific height,
\begin{equation}
    \phi_a(t) = \phi_a(0)e^{-kt}\,.
    \label{eq:specconc}
\end{equation}

\section{Experimental Methods}

\subsection{Materials}

Our experiment was conducted in a recirculating flume. Water was propelled by an electric pump through a pipe of gradually increasing hydraulic diameter with a rectangular cross-section, through a honeycomb flow collimator, into a rectangular open channel containing our test section, then through another similar pipe feeding back to the pump (Figure \ref{fig:floorplan}). In the upper part of our pump's span (span: 0-100 Hz), cavitation (i.e., taking on bubbles of air) occurred, which was accompanied by a distinct noise. This leads to greater variance in the pump's volumetric output, which translates to uncontrolled irregularity in flow velocity in the test section. To eliminate this as a confounding factor, we found the point at which it began by gradually increasing the pump's velocity while monitoring for sound. We determined that the pump ran quietly and presumably free of cavitation up to a rotation frequency greater than 30 Hz, so this value which was chosen as the maximum for our experimental treatments.

\begin{figure}[htbp]
\includegraphics[width=15cm]{pics/flume_with_sedtraps.png}
\centering
\caption{Conceptual diagram of the laboratory flume. Labeled parts are: 1) pump, 2) magnetic flowmeter, 3) test section, and 4) honeycomb flow collimator. Arrows indicate direction of flow. A side view of the open-channel part of the system is included. Some dimensions not to scale. Red circles in the top-down view and black rectangles in the side view of the test section show the locations of the sediment traps.}
\label{fig:floorplan}
\end{figure}

We used 1/8" wooden dowels ($d_c = \SI{.3175}{\centi\metre}$) as collector stems, which were installed in a removable array that filled a recessed well in the bottom of the rectangular channel. The perforated top of the array was covered with aluminum foil, and the gap between the array and the upstream edge of the well was covered with an attached, flat, rectangular piece of aluminum in order to streamline the channel. 

Holes measuring approximately 2.5 cm in diameter were drilled in the array to hold the sediment traps, which were spaced uniformly throughout the test section in order to estimate the mass of sediment settled due to gravity. Each trap consisted of a hollow plastic cylinder with a small perforated disk affixed to the inside, on which was placed a glass microfiber filter paper to capture sediment. Nine sediment traps were used, configured in a shape approximating a $3 \times 3$ grid (Figure \ref{fig:floorplan}).

We chose to use granular walnut shell flour to simulate sediment for our experiments because of the availability of homogeneous, sieve-measured grain sizes in suitable quantities. We performed preliminary experiments with a range of grain sizes but settled on WF5-200 as an appropriate match for sediment sizes in natural floodplains. We measured the distribution of particle size using a LISST-Portable XR laser interferometer (Sequoia Scientific Inc, Bellevue, WA). Median particle size was determined to be approximately \SI{25}{\micro\metre} (Figure \ref{fig:sedsize}).

\begin{figure}[htbp]
\includegraphics[width=15cm] {wf5-200sizedist.png}
\centering
\caption{Particle size distribution of WF5-200 walnut shell flour used as sediment in experiments, expressed as a probability density function (left) and cumulative density function (right). The mode of the distribution (left) and the 50th and 84th percentiles (right) are labeled.} 
\label{fig:sedsize}
\end{figure}

\subsection{Particle Concentration Run Protocols}

In order to quantify the suspended particle concentration, we sampled water from the recirculating flume at regular intervals of 300 seconds throughout the duration of each experiment. Two peristaltic pumps were placed upstream and downstream of our test section, and these pumps were attached through hoses to three laterally-centered, upstream-facing inlets at heights of 5, 14, and 27 centimeters above the flume bed. The water samples from the flume were collected in plastic sampling bottles, each with around 140 mL volume capacity. The plastic bottles were cleaned between each experiment. 

We then processed the samples using vacuum filtration. Before filtration, glass microfiber filter papers were oven-dried for 24 hours at 40 degrees Celsius to remove any moisture, weighed, and assigned to a flume sample. Each water sample was run through its matching filter paper using a vacuum pump, dried once more for 24 hours, and then weighed. The resulting mass of the sediment on each filter paper was divided by the volume of its respective water sample in order to calculate the mass concentration.

The model described in section 2.2 includes only the effects of interest, which take place solely in the test section. Control runs with no collectors ($k_c = 0$) are necessary to account for the "background" decay in suspended sediment concentration due to settling in the rest of the flume ($k_{bg}$). Settling still occurs in the test section during these runs, so it was measured as usual and appears in the calculation $k_{bg} = k - k_s$. Because this rate may depend on Reynolds number \citep{Nielsen_1993, Jacobs_2016, Wang_2018}, control runs were done for each treatment level. Collector density in the test section was assumed not to affect settling outside the test section, so no duplication was performed for this parameter.

%Additionally, runs were required to estimate the entrainment rate. These were performed by adding sediment with the flume running as usual to distribute it uniformly, then turning the flume off and letting all sediment settle. At this point, a sample blank was taken. Then, the flume was turned back on and set to 30 Hz, corresponding to the highest Reynolds number of our experimental runs. 
%Not sure whether we'll dig into the constituent parts of k_s but if so, we might want to consider using the LISST since it presumably has a far lower detection limit

\subsection{Turbulence Characterization Protocols}
Continuous velocity measurements were taken with an acoustic Doppler velocimeter (ADV) at all points within our parameter space. At each point, velocity measurements were taken at 6.7, 14.8, 24.9 and 36.1 cm above the flume bed, with each measurement lasting for at least 300 seconds. The ADV measured velocity, along with other data such as autocorrelation and signal-to-noise ratio, at a frequency of 10 Hz in the $x$, $y$, and two independent measures of data in the $z$ direction. Significant differences between these two measurements help inform data quality during post processing. The instrument was positioned within the flume so that with respect to flow, $x$ denoted the longitudinal direction, $y$ the transverse direction, and $z$ the vertical direction. Before each measurement, a mixture of walnut shell flour and tap water was released into the running flume, as the particulates were needed to act as a tracer in order for the ADV to accurately detect flow velocity.
%TODO

\section{Statistical Analysis}

\subsection{Settling and Capture}
The mass of sediment settled due to gravity was estimated by weighing oven-dried sediment trap microfiber filter papers before and after each experiment and scaling the mass of sediment by the area of the sediment trap. As there were no significant differences between sediment mass collected laterally and longitudinally with respect to flow throughout the test section, the mass settled per sediment trap area was then averaged across all nine traps. Settling outside of the test section was calculated using this process on sediment traps taken from only our control experiments in which no dowels were installed, and scaling the average mass settled per trap area by the area of the flume without the test section ($1.98\, m^{2}$). Settling inside the test section was calculated using the same process on sediment traps taken from each experiment with dowels, and scaling the average mass settled per trap area by the area of the test section ($1.17\, m^{2}$). Total mass of sediment settled throughout the flume is the sum of settling inside and outside the test section.
% i'm not sure if this should go here. help!! also something about rate of settling should probably be here

\subsection{Turbulence and Bed Shear Stress}
Measurements taken by the ADV were filtered and despiked. Points in the time series that had low autocorrelation and points where there were large discrepancies between the two velocity measurements in the $z$ direction were excluded from analysis. Furthermore, we utilized the threshold ADV despiking algorithm detailed in Goring and Nikora (2002) in order to detect spikes in the data, as well as replace these spikes using cubic interpolation.
%need to cite this

Turbulence kinetic energy (TKE) was then calculated from the cleaned data using the equation 
\[\frac{1}{2}(<u^{'2}> + <v^{'2}> + <w^{'2}>),\]
in which, with respect to flow, $u'$ denotes velocity fluctuations in the longitudinal direction, $v'$ in the transverse direction, and $w'$ in the vertical direction. This calculation was done for each of the four heights we took measurements at in the flume, and for each Reynolds number and collector density in our parameter space. Because the velocimeter recorded three measurements simultaneously at each timepoint, TKE was calculated separately for these three measurements. 

Turbulence calculated only from the bottom-most velocity measurements taken by the ADV was multiplied by a proportionality constant $C_{1}$ (0.19) and the density of water $\rho$ in order to estimate bed shear stress ($\tau_{0}$).
\[\tau_{0} = C_{1}[\frac{1}{2}\rho(<u^{'2}> + <v^{'2}> + <w^{'2}>)]\]
Reynolds stress and a modified TKE, in which only fluctuations in the vertical direction with respect to flow are taken into account, can also be used to estimate bed shear stress; these three methods of generating bed shear stress profiles are described in Biron et al. (2004), and the method that produced the smoothest profiles, in our case bed shear stress estimated using TKE, was ultimately chosen.  


\chapter{Results}

\begin{figure}[htbp]
\includegraphics[width=6in]{etafig.png}
\centering
\caption{A plot showing our calculated effective collector efficiency ($\eta^\prime$) values for various values of frontal area ($I_cd_c$) and $\Rey_c$}
\label{fig:eta}
\end{figure}

\begin{figure}[htbp]
\includegraphics[width=6in]{vectrino.png}
\centering
\caption{Flow velocity profiles along the vertical axis from the bed to the water surface of the open rectangular channel for each of our collector density and Reynolds number treatments}
\label{fig:vectrino}
\end{figure}



%\chapter{Discussion}
%
%TODO
%
%The assumption that probability of retention ($p_r$) is 100\% when stems are coated with grease might be questionable based on our result that biofilm increased $\eta^\prime$ in comparison to greased dowels. The only other effect of biofilm that is readily apparent to me is perhaps increasing effective stem diameter ($d_c$). However, with our model for Reynolds number ($\Rey_c \propto d_c$), and both us and \cite{Fauria_2015} finding $\eta^\prime \sim -\Rey_c$, increasing stem diameter would be expected to have the opposite effect, unless I'm missing something.

\bibliographystyle{apalike}
\bibliography{refs}

\end{document}
