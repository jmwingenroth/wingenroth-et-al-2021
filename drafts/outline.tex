% This file should be compiled as such: PdfLatex -> Bibtex -> PdfLatex (x2)

\documentclass{scrreprt}

\KOMAoptions{numbers=noendperiod}

\usepackage{chngcntr}
\counterwithout{figure}{chapter}

\usepackage[utf8]{inputenc}

\usepackage{amsmath}
\newcommand\Rey{\mathrm{Re}}

\usepackage{amsfonts}
\usepackage{amssymb}
\usepackage{outlines}
\usepackage{xcolor}
\usepackage{geometry}
\usepackage{siunitx}

\usepackage{marginnote}
\renewcommand*{\marginfont}{\color{red}\sffamily}

\usepackage{lineno}
\renewcommand\linenumberfont{\normalfont\small\sffamily}
\linenumbers

\usepackage{graphicx}
\graphicspath{{../pics/}}

\usepackage{natbib}

\author{Jordan Wingenroth, Candace Yee, Justin Nghiem, and Laurel Larsen}

\title{Effective collector efficiency of sparse and dense arrays of emergent cylindrical collectors in laminar-turbulent transitional flows}

\begin{document}

\maketitle

%\chapter{Introduction}
%
%TODO
%
%\chapter{Results}
%
%TODO
%
%\chapter{Discussion}
%
%TODO
%
%The assumption that probability of retention ($p_r$) is 100\% when stems are coated with grease might be questionable based on our result that biofilm increased $\eta\prime$ in comparison to greased dowels. The only other effect of biofilm that is readily apparent to me is perhaps increasing effective stem diameter ($d_c$). However, with our model for Reynolds number ($\Rey_c \propto d_c$), and both us and \cite{Fauria_2015} finding $\eta\prime \sim -\Rey_c$, increasing stem diameter would be expected to have the opposite effect, unless I'm missing something.

\setcounter{chapter}{1}

\chapter{Methods}

\section{Theoretical Background}

In this study, we focus on direct interception of suspended particles by emergent vegetation (i.e., collectors). Direct interception is the adhesion of particles traveling along streamlines to the surfaces of collectors (e.g., plants). This is just one of several fluid dynamics mechanisms that can remove particles from suspension. Others include gravitational settling, inertial impaction, and diffusional deposition. Gravitational settling is the process of particles denser than water being drawn towards and coming to rest on the channel bottom. Inertial impaction and diffusional deposition are other means by which suspended particles can adhere to collectors. Inertial impaction occurs when particles possess momentum relative to streamlines, and diffusional deposition arises due to some combination of turbulence and Brownian motion.

Capture efficiency ($\eta=\frac{b}{d_c}$) is defined as the ratio between the upstream width ($b$) of the streamlines that encounter a collector and the collector's own diameter ($d_c$) (Figure \ref{fig:capeff}). Effective capture efficiency ($\eta\prime=p_r\frac{b}{d_c}$) takes into account the fact that not all particles coming into contact with a collector will adhere to it, thus including probability of retention ($p_r$) as a multiplicand in its calculation. This parameter, as opposed to theoretical capture efficiency $\eta$, corresponds to the empirical estimates made by our study and presumably any others that involve $p_r$ as an implicit factor.

\begin{figure}[htbp]
\includegraphics[width=10cm]{collectorefficiency.png}
\centering
\caption{A diagram illustrating capture efficiency for a cylindrical collector. $b$ is the horizontal width of upstream flow and $d_c$ is collector diameter. Adapted from \cite{Palmer_2004}.}
\label{fig:capeff}
\end{figure}

Particle capture in transitionally turbulent flows ($1<\Rey<1000$), which occur commonly in the environments and on the length scales of aquatic macrovegetation, does not follow the analytical expressions involving Reynolds number, collector diameter, and particle diameter ($d_p$), derived for creeping flow. Hence, empirical estimates must be used. \cite{Palmer_2004} introduced a power law expression for estimating capture efficiency of cylindrical collectors, which takes the form \[\eta=C{\Rey_c}^{a}R^{b}\,,\] where collector Reynolds number $\Rey_c=\frac{ud_c}{\nu}$, $u$ is flow velocity, $\nu$ is kinematic velocity, and $R=\frac{d_c}{d_p}$ is the ratio between collector diameter and particle diameter ($d_p$). $C$, $a$, and $b$ are empirically determined regression coefficients. It is worth noting that \cite{Palmer_2004} used $\eta$ in their notation for this model as opposed to $\eta\prime$ because their experiment involved an isolated single cylinder, coated with grease for which retention was assumed to be complete ($p_r = 1$).

As is the case with capture efficiency, turbulence is also theorized and empirically confirmed to affect the settling velocity of heavy, fine particles \citep{Nielsen_1993, Jacobs_2016, Wang_2018}. Increasing turbulence is is one mechanism by which collectors could affect settling rates.

\section{Suspended Particle Concentration Model}

We used a model adapted from \cite{Fauria_2015} to estimate collection efficiency based on exponential decay of suspended sediment concentration, after introducing sediment in a single, initial dose. The key equation in the exponential decay model is \[\frac{d\bar{\phi}}{dt} = -[\frac{Cv_s}{h}(1-E_r) + \eta^{\prime}ud_cI_c]\bar{\phi} = -k\bar{\phi}\,.\] The two summands in brackets represent the net decrease of average suspended settlement concentration ($\bar{\phi}$) due to settling and collection respectively. $C$ is a constant based on the shape of the vertical sediment profile, which is assumed to follow the Rouse equation; $v_s$ is the settling velocity; $h$ is water depth, which equals the collector height for emergent collectors; and $E_r$ is the rate of entrainment (i.e., uptake of particle from the channel bed).

\section{Experimental Methods}

\subsection{Materials}

Our experiment was conducted in a recirculating flume. Water was propelled by an electric pump through a pipe of gradually increasing hydraulic diameter with a rectangular cross-section, through a honeycomb flow collimator, into a rectangular open channel containing our test section, then through another similar pipe feeding back to the pump (Figure \ref{fig:floorplan}). In the upper part of our pump's span (span: 0-100 Hz), cavitation (i.e., taking on bubbles of air) occurred, which was accompanied by a distinct noise. This leads to greater variance in the pump's volumetric output, which translates to uncontrolled irregularity in flow velocity in the test section. To eliminate this as a confounding factor, we found the point at which it began by gradually increasing the pump's velocity while monitoring for sound. We determined that the pump ran quietly and presumably free of cavitation up to a rotation frequency greater than 30 Hz, so this value which was chosen as the maximum for our experimental treatments.

\begin{figure}[htbp]
TODO
\centering
\caption{flume floorplan}
\label{fig:floorplan}
\end{figure}

We used 1/8" wooden dowels ($d_c = \SI{.3175}{\centi\metre}$) as collector stems, which were installed in a removable array that filled a recessed well in the bottom of the rectangular channel. The perforated top of the array was covered with aluminum foil, and the gap between the array and the upstream edge of the well was covered with an attached, flat, rectangular piece of aluminum in order to avoid irregularities in the channel's cross-section. 

Holes measuring approximately 2.5 cm in diameter were drilled in the array to hold the sediment traps. %TODO sed trap paragraph

We chose to use granular walnut shell flour to simulate sediment for our experiments because of the availability of homogeneous, sieve-measured grain sizes in suitable quantities. We performed preliminary experiments with a range of grain sizes but settled on WF5-200 as an appropriate match for sediment sizes in natural floodplains. We measured the distribution of particle size using a LISST-Portable XR laser interferometer (Sequoia Scientific Inc, Bellevue, WA). Median particle size was determined to be approximately \SI{25}{\micro\metre} (Figure \ref{fig:sedsize}).

\begin{figure}[htbp]
\includegraphics[width=15cm,trim = {0 3cm 0 0},clip]{wf5-200sizedist.png}
\centering
\caption{Particle size distribution of WF5-200 walnut shell flour used as sediment in experiments, expressed as a probability density function (left) and cumulative density function (right). The mode of the distribution (left) and the 50th and 84th percentiles (right) are labeled.} %TODO get rid of red X's and make a clean, high-res version of figure
\label{fig:sedsize}
\end{figure}

\subsection{Particle Concentration Run Protocols}

In order to quantify the suspended particle concentration, we sampled water from the recirculating flume at regular intervals of 300 seconds throughout the duration of each experiment. Two peristaltic pumps were placed upstream and downstream of our test section, and these pumps were attached through hoses to three laterally-centered, upstream-facing inlets at heights of 5, 14, and 27 centimeters above the flume bed. The water samples from the flume were collected in plastic sampling bottles, each with around 140 mL volume capacity. The plastic bottles were cleaned between each experiment. 

We then processed the samples using vacuum filtration. Before filtration, glass microfiber filter papers were oven-dried for 24 hours at 40 degrees Celsius to remove any moisture, weighed, and assigned to a flume sample. Each water sample was run through its matching filter paper using a vacuum pump, dried once more for 24 hours, and then weighed. The resulting mass of the sediment on each filter paper was divided by the volume of its respective water sample in order to calculate the mass concentration.

\subsection{Flow Characterization Run Protocols}

%TODO

\section{Statistical Analysis}

%TODO

\bibliographystyle{apalike}
\bibliography{refs}

\end{document}
