% If you would like to post an early version of this manuscript as a preprint, you may use preprint as the journal and change 'submit' to 'accept'. The document class line would be, e.g., \documentclass[preprints,article,accept,moreauthors,pdftex]{mdpi}. This is especially recommended for submission to arXiv, where line numbers should be removed before posting. For preprints.org, the editorial staff will make this change immediately prior to posting.

%---------
% article
%---------
% The default type of manuscript is "article", but can be replaced by: 
% abstract, addendum, article, benchmark, book, bookreview, briefreport, casereport, changes, comment, commentary, communication, conceptpaper, conferenceproceedings, correction, conferencereport, expressionofconcern, extendedabstract, meetingreport, creative, datadescriptor, discussion, editorial, essay, erratum, hypothesis, interestingimages, letter, meetingreport, newbookreceived, obituary, opinion, projectreport, reply, retraction, review, perspective, protocol, shortnote, supfile, technicalnote, viewpoint
% supfile = supplementary materials

%----------
% submit
%----------
% The class option "submit" will be changed to "accept" by the Editorial Office when the paper is accepted. This will only make changes to the frontpage (e.g., the logo of the journal will get visible), the headings, and the copyright information. Also, line numbering will be removed. Journal info and pagination for accepted papers will also be assigned by the Editorial Office.

%------------------
% moreauthors
%------------------
% If there is only one author the class option oneauthor should be used. Otherwise use the class option moreauthors.

%---------
% pdftex
%---------
% The option pdftex is for use with pdfLaTeX. If eps figures are used, remove the option pdftex and use LaTeX and dvi2pdf.

%=================================================================
\firstpage{1} 
\makeatletter 
\setcounter{page}{\@firstpage} 
\makeatother
\pubvolume{TBDXX}
\issuenum{TBDYY}
\articlenumber{TBDZZ}
\pubyear{2021}
\copyrightyear{2021}
\externaleditor{Academic Editor: Dr. Rafael Tinoco}
\history{Received: date; Accepted: date; Published: date}
%\updates{yes} % If there is an update available, un-comment this line

%% MDPI internal command: uncomment if new journal that already uses continuous page numbers 
%\continuouspages{yes}

%------------------------------------------------------------------
% The following line should be uncommented if the LaTeX file is uploaded to arXiv.org
%\pdfoutput=1

%=================================================================
% Add packages and commands here. The following packages are loaded in our class file: fontenc, calc, indentfirst, fancyhdr, graphicx, lastpage, ifthen, lineno, float, amsmath, setspace, enumitem, mathpazo, booktabs, titlesec, etoolbox, amsthm, hyphenat, natbib, hyperref, footmisc, geometry, caption, url, mdframed, tabto, soul, multirow, microtype, tikz

\newcommand\Rey{\mathrm{Re}}

\usepackage{siunitx}
\DeclareSIUnit\year{yr}

\usepackage{textcomp}

\usepackage{threeparttable}
\usepackage{longtable}

%=================================================================
%% Please use the following mathematics environments: Theorem, Lemma, Corollary, Proposition, Characterization, Property, Problem, Example, ExamplesandDefinitions, Hypothesis, Remark, Definition, Notation, Assumption
%% For proofs, please use the proof environment (the amsthm package is loaded by the MDPI class).

%=================================================================
% Full title of the paper (Capitalized)
\Title{Sediment Interception by Emergent Stems Across Varying Patch Densities and Flows}

% Author Orchid ID: enter ID or remove command
\newcommand{\orcidauthorA}{0000-0002-7970-841X} % Add \orcidA{} behind the author's name
%\newcommand{\orcidauthorB}{0000-0000-000-000X} % Add \orcidB{} behind the author's name

% Authors, for the paper (add full first names)
\Author{Jordan Wingenroth $^{1}$*\orcidA{}, Candace Yee $^{2}$, Justin Nghiem$^{1,3\dagger}$ and Laurel Larsen $^{1,2}$}

% Authors, for metadata in PDF
\AuthorNames{Jordan Wingenroth, Candace Yee, Justin Nghiem and Laurel Larsen}

% Affiliations / Addresses (Add [1] after \address if there is only one affiliation.)
\address{%
$^{1}$ \quad Department of Geography, University of California, Berkeley, CA

$^{2}$ \quad Department of Civil and Environmental Engineering, University of California, Berkeley, CA

$^{3}$ \quad Department of Statistics, University of California, Berkeley, CA}

% Contact information of the corresponding author
\corres{Correspondence: j.wingenroth@berkeley.edu}

% Current address and/or shared authorship
\firstnote{Current address: California Institute of Technology, Pasadena, CA} 
%\secondnote{These authors contributed equally to this work.}
% The commands \thirdnote{} till \eighthnote{} are available for further notes

%\simplesumm{} % Simple summary

%\conference{} % An extended version of a conference paper

%TODO
% Abstract (Do not insert blank lines, i.e. \\) 
\abstract{Suspended sediment collected by vegetation in marshes and wetlands contributes to vertical accretion, an important factor in these habitats' futures as sea level rise progresses. Effective capture efficiency (ECE) is a key variable in determining the significance of direct interception in elevation-change models, and one that is not yet thoroughly understood in transitionally turbulent flows. Here we used laboratory flume experiments to determine that ECE decreases with increasing collector Reynolds number (study range: 66 to 200; p < 0.05 for 2 of 3 treatments) and collector density (solid volume fraction: 0.22\% to 1.17\%; p < 0.05 for 2 of 3 treatments), and that biofilm has a considerable positive effect on ECE. This is in agreement with previous similar studies. By combining our data with those of the most similar study, we also present a preliminary model quantitatively assessing the effect of collector density on ECE.}

% Keywords
\keyword{sediment transport; collector efficiency; submerged vegetation; transitional turbulence; biofilm; sedimentation}
